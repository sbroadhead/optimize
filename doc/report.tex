\documentclass{report}
\usepackage[margin=1.25in]{geometry}
\usepackage[sc]{mathpazo}
\linespread{1.05}
\usepackage[T1]{fontenc}

\usepackage{nicefrac}
\usepackage{amsmath}
\usepackage{amssymb}

\begin{document}
\title{Computing a Feasible Trajectory Through an Obstacle Field Using Nonlinear Constrained Optimization}
\author{
    Simon Broadhead\\
    Jeremy Ciesinski\\
    J.D. Cumpson\\
    Craig Skelsey\\
    Peter Smyth\\
    Junaid Syed\\
    Subramanian Venkatesan
}
\maketitle

%%%%%
\begin{abstract}
Abstract goes here
\end{abstract}

%%%%%
\chapter{Problem Overview}

%%%%%

\chapter{Finding a Trajectory in a Convex Region}
\section{Power Series Approximation to the Direction}
The direction component of the movement is given by a pair of
coupled differential equations
\begin{equation} \label{eq:orig}
    \begin{aligned}
    d_x'(t) &= -d_y(t)(t(l-r)+\omega(t)) C\\
    d_y'(t) &= \phantom{-}d_x(t)(t(l-r)+\omega(t)) C
    \end{aligned}
\end{equation}
where $l$ and $r$ are constant and $C = 2/MR$. $\omega(t)$ is the angular velocity at time $t$, and
luckily for us, $\omega'$ is a constant, so we can approximate
the solution to this differential equation using a Maclaurin series.
\begin{equation} \label{eq:series}
    \begin{aligned}
    d_x(t) &= \sum_{n=0}^\infty a_n t^n \\
    d_y(t) &= \sum_{n=0}^\infty b_n t^n
    \end{aligned}
\end{equation}
Substituting Equation \ref{eq:series} into Equation \ref{eq:orig}
(taking the derivative on the left-hand side) gives us
\begin{equation} \label{eq:subst}
    \begin{aligned}
    \sum_{n=0}^\infty n a_n t^{n-1} &=
        -\Bigg[\sum_{n=0}^\infty b_n t^n \Bigg]
        (t(l-r)+\omega(t)) C \\
    \sum_{n=0}^\infty n b_n t^{n-1} &=
        \phantom{-}\Bigg[\sum_{n=0}^\infty a_n t^n \Bigg]
        (t(l-r)+\omega(t)) C
    \end{aligned}
\end{equation}
Letting $t=0$ in Equation \ref{eq:orig} lets us solve for the constant coefficients. $d_x(0)$ and $d_y(0)$ are given,
so $a_0$ and $b_0$ are fully determined.
\begin{equation} \label{eq:coeff0}
    \begin{aligned}
        a_0 &= d_x(0) \\
        b_0 &= d_y(0)
    \end{aligned}
\end{equation}
Letting $t=0$ in
Equation \ref{eq:subst} lets us solve for the linear coefficients. $\omega(0)$ is given, so $a_1$ and $b_1$ are also
fully determined.
\begin{equation} \label{eq:coeff1}
    \begin{aligned}
    a_1 &= -b_0\omega(0)C \\
    b_1 &= \phantom{-}a_0\omega(0)C
    \end{aligned}
\end{equation}

At this point, we want to find a general recurrence for \emph{all} of the
coefficients of the Maclaurin series. That way, we can take as many
terms as we need to achieve the desired accuracy without having to
compute and solve for each coefficient by hand.

Since the $k$th coefficient of the Maclaurin series is given by $f^{(k)}(0)/n!$, we need to find a convenient expression for the $k$th derivative of
Equation \ref{eq:subst}.
First we'll compute the first derivative of Equation \ref{eq:subst}.
\begin{equation} \label{eq:diff1}
    \begin{aligned}
    \sum_{n=0}^\infty n(n-1)a_n t^{n-2} &=
        -\Bigg(\Bigg[\sum_{n=0}^\infty nb_n t^{n-1} \Bigg]
            (t(l-r)+\omega(t))
                + \Bigg[\sum_{n=0}^\infty b_n t^n \Bigg]
                (l - r + \omega')\Bigg)C\\
    \sum_{n=0}^\infty n(n-1)b_n t^{n-2} &=
        \phantom{-}\Bigg(\Bigg[\sum_{n=0}^\infty na_n t^{n-1} \Bigg]
            (t(l-r)+\omega(t))
                + \Bigg[\sum_{n=0}^\infty a_n t^n \Bigg]
                (l - r + \omega')\Bigg)C\\
    \end{aligned}
\end{equation}
Next, we'll compute the second derivative of Equation \ref{eq:subst}.
\begin{equation} \label{eq:diff2}
    \begin{gathered}
    \shoveleft{\sum_{n=0}^\infty n(n-1)(n-2)a_n t^{n-3} =}\\
        \qquad\qquad-\Bigg(\Bigg[\sum_{n=0}^\infty n(n-1)b_n t^{n-2} \Bigg]
             (t(l-r)+\omega(t))
                + 2\Bigg[\sum_{n=0}^\infty n b_n t^{n-1} \Bigg]
                (l - r + \omega')\Bigg)C\\
    \shoveleft{\sum_{n=0}^\infty n(n-1)(n-2)b_n t^{n-3} =}\\
        \qquad\qquad\phantom{-}\Bigg(\Bigg[\sum_{n=0}^\infty n(n-1)a_n t^{n-2} \Bigg]
            (t(l-r)+\omega(t))
                + 2\Bigg[\sum_{n=0}^\infty n a_n t^{n-1} \Bigg]
                (l - r + \omega')\Bigg)C\\
    \end{gathered}
\end{equation}
We can see a pattern emerge if we continue to take derivatives.
We can then represent the $k$th derivative of Equation \ref{eq:subst} as
\begin{equation} \label{eq:diffk}
    \begin{aligned}
    \sum_{n=0}^\infty \frac{n!}{(n-k)!} a_n t^{n-k} =
        -\Bigg(\Bigg[&\sum_{n=0}^\infty \frac{n!}{(n-k+1)!} b_n t^{n-k+1} \Bigg]
             (t(l-r)+\omega(t))
                \\
                +k\Bigg[&\sum_{n=0}^\infty \frac{n!}{(n-k+2)!} b_n t^{n-k+2} \Bigg]
                (l - r + \omega')\Bigg)C\\
    \sum_{n=0}^\infty \frac{n!}{(n-k)!} b_n t^{n-k} =
        \phantom{-}\Bigg(\Bigg[&\sum_{n=0}^\infty \frac{n!}{(n-k+1)!} a_n t^{n-k+1} \Bigg]
             (t(l-r)+\omega(t))
                \\
                + k\Bigg[&\sum_{n=0}^\infty \frac{n!}{(n-k+2)!} a_n t^{n-k+2} \Bigg]
                (l - r + \omega')\Bigg)C.\\
    \end{aligned}
\end{equation}
Now we can find the $k$th coefficient by
\begin{equation}
    \begin{aligned}
        a_k &= d_x^{(k)}(0)/n! \\
            &= -\big(b_{k-1} \omega(0) - k b_{k-2} (l - r + \omega')\big)C,\\
        b_k &= d_y^{(k)}(0)/n! \\
            &= \phantom{-}\big(a_{k-1} \omega(0) + k a_{k-2} (l - r + \omega')\big)C.
    \end{aligned}
\end{equation}
\end{document}